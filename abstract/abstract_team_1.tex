\documentclass[a4paper]{article}

\usepackage[utf8]{inputenc}
\usepackage[UKenglish]{babel}
\usepackage[T1]{fontenc}
\usepackage{graphicx}
\usepackage{url}
\usepackage{hyperref}
\usepackage{blindtext}
\usepackage[top=3.5cm,bottom=3.5cm,left=2.5cm,right=2.5cm]{geometry}
\usepackage{fancyhdr}

\pagestyle{fancy}
\fancyhf{}
\rhead{pro4E Team 1}
\lhead{FHNW}

\title{Abstract}

\begin{document}

\section*{Abstract}

% Motivation
People go to a museum to learn something.
The most natural way of learning from an exhibition is from a fellow human, but guided tours don't allow visitors to explore the exhibition at their own pace.
The next best thing would be an audio guide that tells the visitor about every object in the exhibition when they see it.
The audio guide Dojo is the next evolution in personal guided tours in museums.
% Problem statement
The Dojo audio guide has a sleek, ergonomic and hygenic case.
A bone conduction transducer enables an immersive museum experience without isolating the visitor form their environment.
Each exhibit is fitted with a Bluetooth beacon which transmits a unique ID, enabling the visitor to learn more about that particular object.
The project aims to create an electronics package that can fit in the Dojo's very long and narrow casing and is able to drive the bone conduction transducer.
% Approach
To meet the design criteria, the Dojo incorporates an ATmega328p microcontroller as CPU and a WTV020 audio chip as output device.
Both fit in the Dojo enclosure and the WTV020 can be easily controlled by the ATmega328p's GPIO, enabling a clean separation of user output and control logic, forgoeing the need for an operating system.
When a visitor buys a ticket the Dojo receives a "key" with all Bluetooth beacons the visitor has access to.
The Dojo will search for the closest beacon with its Bluetooth module and if it is close enough it will notify the visitor with a vibration motor.
Should the visitor enter an exhibition to which they don't have access, the Dojo will vibrate and a red LED lights up.
% Results
In tests the Dojo can identify the closest of \texttt{Max\_BEACON\_NUMBER} beacons and can determine its exact distance with precision of \texttt{BLUETOOTH\_PRECISION} meters.
It can store \texttt{TOTAL\_AUDIO} minutes of Audio and up to 255 unique Beacon IDs and a complete key of 255 beacons can be written to the EEPROM in 6.17 seconds.
The battery works \texttt{BATTERY\_LIFE} before the deep discharge protection shuts the Dojo down.
% conclusions
The Dojo has shown itself to be a promising product, enhancing a visit to a museum.

\paragraph{Keywords:} Audio guide, Bluetooth, Bone conduction transducer, ATmega328p, WTV020, HM11

\end{document}
