\documentclass[a4paper]{article}

\usepackage[utf8]{inputenc}
\usepackage[UKenglish]{babel}
\usepackage[T1]{fontenc}
\usepackage{graphicx}
\usepackage{url}
\usepackage{hyperref}
\usepackage{blindtext}
\usepackage[top=3.5cm,bottom=3.5cm,left=2.5cm,right=2.5cm]{geometry}

\begin{document}

\paragraph{Motivation}
When visiting a museum, the exhibitions context is important.
The most natural way of learning about an exhibition is by bringing a guide along, 
	but that is not feasible for most people.
The next best thing is an audio guide that tells the visitor about every object in the exhibition.
It might not be interactive, but every visitor can experience the exhibition at their own pace.

While conventional audio guides are a big improvement over plain text plaques,
	the Dojo is the next evolution in personal guided tours in museums.

\paragraph{Problem statement}
The Dojo audio guide possesses a sleek case that is ergonomic and sanitary.
It employs a bone conduction transducer to enable an immersive museum experience 
	without isolating the visitor form their environment.
Each exhibition object is fitted with a Bluetooth beacon transmitting a unique ID, 
	enabling the visitor to learn more about that object.

The challenge lies in creating an electronics package that fits in the Dojos very long and narrow casing
	and driving the bone conduction transducer.

\paragraph{Approach}
To meet the design criteria, the design incorporates a ATmega328p microcontroller as CPU
	and a WTV020 audio chip as output device.
Both fit in the Dojo enclosure and the WTV020 can be easily controlled by the ATmega328p's GPIO, 
	enabling a clean separation of user output and control logic.
The Dojo's CPU finds the nearest exhibition object via a Bluetooth module connected over UART 
	and will notify the user of additional information on the object they are presumably looking at.

\paragraph{Results}
In tests the Dojo has shown it can identify the closest of \texttt{BEACON\_NUMBER} beacons
	and can determine its exact distance with precision of \texttt{BLUETOOTH\_PRECISION} meters.
It can store \texttt{TOTAL\_AUDIO} minutes of Audio and up to 255 unique Beacon IDs.

For a typical use case, the battery has been shown to last \texttt{BATTERY\_LIFE}
	before the deep discharge protection shuts the Dojo down.

\paragraph{conclusions}
The Dojo has shown itself to be a viable product, increasing the quality of a museum visitors experience.

\end{document}
