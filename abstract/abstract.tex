\documentclass[a4paper]{article}

\usepackage[utf8]{inputenc}
\usepackage[UKenglish]{babel}
\usepackage[T1]{fontenc}
\usepackage{graphicx}
\usepackage{url}
\usepackage{hyperref}
\usepackage{blindtext}
\usepackage[top=3.5cm,bottom=3.5cm,left=2.5cm,right=2.5cm]{geometry}
\usepackage{fancyhdr}

\pagestyle{fancy}
\fancyhf{}
\rhead{pro4E Team 1}
\lhead{FHNW}

\title{Abstract}

\begin{document}

\section*{Abstract}

% Motivation
When visiting a museum, the exhibitions context is important.
The most natural way of learning about an exhibition is from a fellow human, 
	but guided tours don't allow visitors to explore the exhibition naturally.
The next best thing is an audio guide that tells the visitor about every object in the exhibition when they discover it.
It might not be interactive, but all visitors can experience the exhibition at their own pace.
While conventional audio guides are a big improvement over plain text plaques,
	the Dojo is the next evolution in personal guided tours in museums.

% Problem statement
The Dojo audio guide possesses a sleek case that is ergonomic and sanitary.
It employs a bone conduction transducer to enable an immersive museum experience 
	without isolating the visitor form their environment.
Each exhibition object is fitted with a Bluetooth beacon transmitting a unique ID, 
	enabling the visitor to learn more about that object.
The challenge lies in creating an electronics package that fits in the Dojos very long and narrow casing
	and fulfilling the bone conduction transducer's demanding power requirements.

% Approach
To meet the design criteria, the Dojo incorporates an ATmega328p microcontroller as CPU
	and a WTV020 audio chip as output device.
Both fit in the Dojo enclosure and the WTV020 can be easily controlled by the ATmega328p's GPIO, 
	enabling a clean separation of user output and control logic, forgoeing the need for an operating system.
In a typical usage scenario the Dojo is loaded with a audio package before opening hours.
When a visitor buys a ticket the Dojo receives a "key" with all Bluetooth beacons they have access to.
After entering the exhibition, the Dojo will search for the closest beacon with its Bluetooth module
	and if it is close enough it will notify the visitor with the vibration motor.
Should the visitor enter an exhibition they don't have access to, 
	the Dojo will notify them of this with the vibration motor and the red LED band.

% Results
In tests the Dojo has shown it can identify the closest of \texttt{BEACON\_NUMBER} beacons
	and can determine its exact distance with precision of \texttt{BLUETOOTH\_PRECISION} meters.
It can store \texttt{TOTAL\_AUDIO} minutes of Audio and up to 255 unique Beacon IDs.
For a typical use case, the battery has been shown to last \texttt{BATTERY\_LIFE}
	before the deep discharge protection shuts the Dojo down.
The current prototype is printed on a double layer circuit board and measures \texttt{DIMENSIONS}, 
	however since it only incorporates components that fit in the Dojo's case, 
	the final product will only require a different circuit layout and a \texttt{PCB\_LAYERS} layer circuit board.

% conclusions
The Dojo has shown itself to be a promising product, increasing the quality of a museum visitors experience.
It is based on commonly available components and can be purchased for an affordable price.

\paragraph{Keywords:} Audio guide, Bluetooth, Bone conduction transducer, ATmega328p, WTV020, HM11

\end{document}
