\chapter{Einleitung}
% ================ Einstellungen =======================
\thispagestyle{fancy} \rhead{\slshape Einleitung} \setcounter{page}{1}
% ======================================================
Museen bieten die Möglichkeit verschiedenste Ausstellung den Interessenten nahe zu bringen. Um dem Besucher zu ermöglichen, sich mit den persönlich relevanten Objekten auseinander zu setzen, werden unter anderem Audio-Guides verwendet. Die Audioguides werden meist mit Hilfe von Standard-Kopfhörern umgesetzt. Diese Kopfhörer müssen aufgrund von Hygienegründen gereinigt werden. Diese Reinigungskosten sind verhältnismässig zu gross. Zudem bieten die Museen nicht die Möglichkeit, eine Ausstellung einzugrenzen. Eine Unterteilung der Ausstellung würde dem Besucher die Option geben, eine Eintrittskarte zu erwerben, welche sich nur auf einen Teil der Ausstellung bezieht. Dies könnte Museumsaustellung attraktiver gestalten.\\

Ziel ist es, einen Audioguide zu entwickeln welcher es ermöglicht eine Ausstellung einzugrenzen, in dem es nur die gewählten Audiodateien abspielt und allenfalls als Zutrittsberechtigung fungiert. Die Audioübertragung soll über einen Knochenschallgeber funktionieren, dies würde die Reinigungskosten senken. Mit Hilfe des Audioguides soll es möglich sein, Sounddateien abzuspielen, sobald man sich in der Nähe eines Kunstobjektes befindet.\\

Mit Hilfe von Bluetooth Beacons können den Kunstobjekten Audiodateien zugeordnet werden. Durch einen Vibrator kann angekündigt werden, dass eine Audiodatei vorhanden ist, welche durch Bestätigung des Besuchers über den Knochenschallgeber abgespielt werden kann.\\

Mit Hilfe des Prototyps ist es möglich, Beacons in einem Abstand von xxx Metern korrekt zu erfassen und die zugeordnete Audiodatei abzuspielen. Auf dem Dojo können YYY Audiodateien abgespeichert werden. Am Eintritt können je nach Wunsch die verschiedenen Audiofiles ausgewählt werden und für den Besuch freigegeben werden. Dem Besucher wird die Möglichkeit gegeben, für gewünschte Objekte mit Hilfe eines Like-Buttons am Ende des Besuches Zusatzinformationen in Form einer Broschüre zu erhalten.\\

Der nachfolgende Bericht ist in sechs Teile aufgeteilt. Durch das Gesamtkonzept wird die Funktion sowie das Prinzip von Dojo erklärt. In den folgenden zwei Kapiteln wird die benötigte Energieversorgung sowie die Software für das Dojo dargelegt. In den letzten drei Kapiteln werden die Schnittstellen sowie die Firmware erläutert. 