\chapter{Einleitung}
% ================ Einstellungen =======================
\thispagestyle{fancy} \rhead{\slshape Einleitung} \setcounter{page}{1}
% ======================================================
Museen haben die Aufgabe, mit Ausstellungen spezifische Themenbereiche den Interessierten zugänglich und erlebbar zu machen. Dabei soll oft ein breites Publikum vom wenig Informierten bis zur Expertin angesprochen werden. Um den Besuchenden zu ermöglichen, sich mit den persönlich relevanten Objekten auseinanderzusetzen, werden unter anderem Audioguides verwendet. Diese Audioguides werden meist zusammen mit Standard-Kopfhörern eingesetzt. Um den Hygienestandards zu entsprechen, müssen die Kopfhörer entweder aufwändig gereinigt oder Einwegkopfhörer eingesetzt werden. Der Kostenaufwand ist verhältnismässig gross, zudem bieten die bisher verwendeten Audioguides keine Möglichkeit, eine Ausstellung aufzuteilen. Eine Unterteilung der Ausstellung würde den Museen neue Angebote ermöglichen, z. B. Eintrittskarten für ausgewählte Ausstellungsbereiche mit entsprechend individualisierten Audioguides anzubieten. Dies könnte eine Museumsausstellung attraktiver gestalten. \\

Ziel dieser Projektarbeit war es, einen Audioguide namens Dojo mit der notwendigen Elektronik auszustatten. Das Konzept Dojo wurde bereits erarbeitet, das heisst im Rahmen dieser Arbeit stand die technische Umsetzung des Konzeptes von Dojo im Zentrum. Dojo soll eine Ausstellung eingrenzen können, indem es nur die gewählten Audiodateien abspielt und allenfalls als Zutrittsberechtigung fungiert.  Die Audioübertragung vom Gerät ans Gehör soll über einen Knochenschallgeber realisiert werden, um die Unterhaltskosten der Geräte signifikant zu senken. Mit Hilfe des Audioguides soll es möglich sein, Sounddateien abzuspielen, sobald man sich in der Nähe eines Ausstellungsobjektes befindet. Dojo soll zudem eine Funktion, welche einem Like-Button ähnlich ist, besitzen. Diese Funktion soll dem Besucher und der Besucherin ermöglichen, Objekte auszuwählen und später mehr Informationen darüber z. B. in Form einer Broschüre zu erhalten. Die zu entwickelnde Elektronik muss in dem bereits bestehenden Design Platz finden. \\

Um die gewünschten Audiodateien freizuschalten und den richtigen Ausstellungsobjekten zuzuordnen, wird am Anfang des Besuches ein Schlüssel auf das Dojo geladen, welcher anhand einer Korrespondenztabelle die ausgewählten Dateien freischaltet. Mit Hilfe von Bluetooth-Beacons, welche im Museum bei den Objekten befestigt sind, können den Ausstellungsobjekten Audiodateien zugeordnet werden. Durch einen Vibrationsmotor wird signalisiert, dass eine Audiodatei abspielbereit ist. Die Audiodatei, welche sich auf einer $\mu$SD-Karte befindet, wird nach Betätigung des Start-Buttons über einen WTV020 Chip auf dem Knochenschallgeber abgespielt. Wenn der Like-Button während dem Rundgang betätigt wird, kann anschliessend am Ausgang über die Korrespondenztabelle eine individuelle Broschüre ausgedruckt werden oder man kann sich die Broschüre als PDF per E-Mail zusenden lassen.\\

Mit Hilfe des hergestellten Prototyps ist es möglich, Beacons in einem Abstand von fünf Metern korrekt zu erfassen und die zugeordnete Audiodatei abzuspielen. Auf dem Dojo können maximal 512 Audiodateien abgespeichert werden. Am Eintritt können je nach Wunsch die verschiedenen Audiofiles ausgewählt werden und für den Besuch freigegeben werden. Den Besuchern und den Besucherinnen wird die Möglichkeit gegeben, für gewünschte Objekte mit Hilfe eines Like-Buttons am Ende des Museumsrundgangs Zusatzinformationen in Form einer Broschüre zu erhalten.\\
 
In dem nachfolgenden Kapitel \nameref{Gesamtsystem}, wird die Aufteilung des Berichtes dargelegt. 
