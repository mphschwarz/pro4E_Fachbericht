\chapter{Software}
% ================ Einstellungen =======================
\thispagestyle{fancy} \rhead{\slshape Software}
% ======================================================
Im Kapitel Software wird erläutert, wie die Audiodateien auf das Dojo gelangen und die Korrespondenztabelle angepasst werden kann. Es wird erklärt, wie die Likes ausgewertet und verarbeitet werden können.
\section{Konzept}
In diesem unter Kapitel wir das erarbeitete Konzept für die Datenübertragung von einem Computer auf das Dojo dargelegt.

\section{Datenverwaltung}
Alle Austellungsobjete sind in einer Textdatei aufgelistet, in der vermerkt wird, welche Audio- und Textdatei zu welchem Beaon gehört.
Zudem wird in einer Weiteren Textdatei ein Preset definiert, mit der die SDKarte beschrieben werden kann.

Der CLI-Befehl nimmt die Inventardatei und die Presetdatei als Argumente und fragt den Nutzer, welche Ausstellungen und Sprache dem Besucher zur Ferfügung stehen sollen. 
Mit diesen Inputs wird ein Ticket generiert, das über die Serielle Schnittstelle auf das EEPROM des Mikrocontrollers geladen wird.

% In diesem Kapitel wird beschrieben wie die Konfigurationsdateien ausgelesen werden. Zudem wird die Korrespondenztabelle erläutert. 

\section{Auslesen, Schreiben}
Hier wird beschrieben wie die Verbindung und die Datenübertragung auf den internen EEPROM realisiert wird.

\section{Validierung}
Für die Validierung der Datenverwaltung wurden Python Unittests eingesetzt, die mit einer Testdatenbank alle Funktionen testen.
Neben den einzelnen Funktionen wird auch ein kompletter Durchlauf mit der Testdatenbank gemacht.

Die EEPROM-Opperationen werden mit einem Arduino Uno-Board getestet, wobei das EEPROM vom PC aus beschrieben und wieder eingelesen wird.
Die Unittest stellen sicher, dass das EEPROM die gewünschten Daten beinhaltet und dass diese mit der spezifizierten Geschwindigkeit übertragen werden.

% Hier werden die Python Unittests, welche verwendet wurden, dargelegt.

