\chapter{Firmware}
\label{Firmware}
% ================ Einstellungen =======================
\thispagestyle{fancy} \rhead{\slshape Firmware}
% ======================================================
\section{Konzept}
\section{Statemachine}
Auf dem Mikrocontroller läuft eine Mealy-Statemachine mit fünf Zuständen:\\
\begin{itemize}[leftmargin=3.2cm]
\item[SCAN:] Es wird vom Dojo nach IBeacons in naher Umgebung gesucht. Falls einer gefunden wurde, dann wird direkt das dazugehörige Audiofile bereitgestellt. Wird der Playbutton gedrückt, ändert sich der State zu PLAY. Ansonsten wird weiter gescanned.
\item[PLAY:] Hier werden die bereitgestellten Audiofiles abgespielt. Wird der Playbutton nochmals gedrückt, wird das Abspielen abgebrochen und der State wechselt wieder zu SCAN.
\end{itemize}
Wird das Dojo an den Computer angeschlossen, ist der Zustand des Dojos abhängig von der gewünschten Tätigkeit. Dafür wird vom Computer aus ein Befehl über das USB-Kabel gesendet, woraufhin sich der State ändert:
\begin{itemize}[leftmargin=3.2cm]
\item[GET\_Likes:] Die vom Besucher getätigten Likes werden vom EEPROM auf den Computer transferiert.
\item[LOAD\_SD:] Die microSD-Karte wird mit den gewünschten Audiofiles beschrieben.
\item[LOAD\_CONFIG:] Ein Ticket wird auf das interne EEPROM geladen.
\end{itemize}
 \section{Datenverwaltung}
%In diesem Kapitel wird die Datenverwaltung auf dem internen EEPROM erklärt.
Das Ticket auf dem EEPROM beinhaltet für jedes Ausstellungsobjekt zwei Bytes, das erste beinhaltet die Beacon-ID und das zweite das Zutritts- und Like-Bit.
Die ersten zwei Bytes sind reserviert und beinhalten die Anzahl gespeicherter Beacons und ein Sprachcode: 0 für die erste Sprace, 1 für die zweite.
Da auf der SD-Karte die Dateien der zwei Sparchen abewechslungsweise und in gleicher Reihenfolge wie im EEPROM aufgelistet sind, muss nur der Index der Beaon-ID übergeben werden damit die korrekte Audiodatei abgespielt werden kann: \\
\texttt{(BeaconIndex - 2) * 2 + Sprachcode}.
\section{Validierung}
%Hier wird die Validierung der kompletten Firmware sowie dessen Ergebnisse dargelegt. 
Die Validierung der Datenverwaltung wird auch mit Unittests vom PC aus gemacht.
Es wird dabei ein mit Zufallszahlengenerator generiertes Ticket über die serielle Schnittstelle auf das EEPROM geschrieben und dann wieder eingelesen.
