\documentclass[a4paper,11pt]{scrreprt}

\usepackage[utf8]{inputenc}
\usepackage[ngerman]{babel}
\usepackage[T1]{fontenc}
\usepackage{amsmath}
\usepackage{graphicx}
\usepackage{xcolor}
\usepackage{wrapfig}
\usepackage{multirow}
\usepackage{ulem}
\usepackage{booktabs}
\usepackage{caption}
\usepackage{subcaption}
\usepackage{url}
\usepackage{fancyhdr}
\usepackage{lscape}
\usepackage{hyperref}
\usepackage{blindtext}
\usepackage{adjustbox}
\usepackage{colortbl}
\usepackage{enumitem}
\usepackage{float}
\usepackage[final]{pdfpages}%Import von PDF
\usepackage[top=3.5cm,bottom=3.5cm,left=2.5cm,right=2.5cm]{geometry}

\bibliographystyle{unsrt}
\parindent0pt


%Kopf-& Fusszeile----------------------------
\pagestyle{fancy}
\lhead{\includegraphics[width=1.5cm]{Bilder/fhnw_logo.png}}
\chead{Elektro- und Informationstechnik}
\renewcommand{\headrulewidth}{0.4pt}
%--------------------------------------------

\renewcommand*\chapterheadstartvskip{\vspace*{-0.2cm}}

\begin{document}
\thispagestyle{empty}

\begin{center}
\begin{tabular}{p{\textwidth}}

\vspace*{-3.5cm}

\begin{flushleft}
\includegraphics[scale=1.3]{Bilder/FHNW.png}
\end{flushleft}

\\

\vspace*{-2cm}
\begin{center}
\textcolor{black}{
\textbf{
\Huge{
Fachbericht
}}}
\end{center}

\\

\begin{center}
\Large{
\textbf{
Team 1 
\\[0.5cm]
Projekt 4 
}}
\end{center}

\\

\begin{center}
\large{
Fachhochschule Nordwestschweiz FHNW
}
\end{center}

\\

\begin{center}
\large{\today}
\end{center}

\begin{center}
\includegraphics[width=5cm]{Bilder/printTitel.png}
\end{center}

\\

\begin{center}
\begin{tabular}{ll}
\toprule 
\textbf{Studiengang:} 		& Elektro- und Informationstechnik EIT \\
\hline
\textbf{Auftraggeber/in:} 	& Prof. Hans Gysin\\
							& Jana Kalbermatter\\
\hline
\textbf{Fachexperten:} 		& Matthias Meier \\
							& Prof. Dr. Pascal Schleuniger \\
							& Pascal Buchschacher \\
							& Dr. Roswitha Dubach \\
							& Dr. Anita Gertiser \\
							& Bonnie Domenghino \\
\hline
\textbf{Projektteam:} & Adrian Annaheim \\ 
							& Benjamin Ebeling \\ 
 							& Jonas Rosenmund\\ 
 							& Michael Schwarz  \\ 
 							& Samuel Wey \\ 
 							& Andres Minder \\ 
\bottomrule
\end{tabular}
\end{center}

\end{tabular}
\end{center}

\cleardoublepage
\section*{Abstract}
%\documentclass[a4paper]{article}
%
%\usepackage[utf8]{inputenc}
%\usepackage[UKenglish]{babel}
%\usepackage[T1]{fontenc}
%\usepackage{graphicx}
%\usepackage{url}
%\usepackage{hyperref}
%\usepackage{blindtext}
%\usepackage[top=3.5cm,bottom=3.5cm,left=2.5cm,right=2.5cm]{geometry}
%\usepackage{fancyhdr}
%
%\pagestyle{fancy}
%\fancyhf{}
%\rhead{pro4E Team 1}
%\lhead{FHNW}
%
%\title{Abstract}
%
%\begin{document}
%
%\section*{Abstract}

% Motivation
People go to a museum to learn something.
The most natural way of learning from an exhibition is from a fellow human, but guided tours don't allow visitors to explore the exhibition at their own pace.
The next best thing would be an audio guide that tells the visitor about every object in the exhibition when they see it.
The audio guide Dojo is the next evolution in personal guided tours in museums.
% Problem statement
The Dojo audio guide has a sleek, ergonomic and hygenic case.
A bone conduction transducer enables an immersive museum experience without isolating the visitor form their environment.
Each exhibit is fitted with a Bluetooth beacon which transmits a unique ID, enabling the visitor to learn more about that particular object.
The project aims to create an electronics package that can fit in the Dojo's very long and narrow casing and is able to drive the bone conduction transducer.
% Approach
To meet the design criteria, the Dojo incorporates an ATmega328p microcontroller as CPU and a WTV020 audio chip as output device.
Both fit in the Dojo enclosure and the WTV020 can be easily controlled by the ATmega328p's GPIO, enabling a clean separation of user output and control logic, forgoeing the need for an operating system.
When a visitor buys a ticket the Dojo receives a "key" with all Bluetooth beacons the visitor has access to.
The Dojo will search for the closest beacon with its Bluetooth module and if it is close enough it will notify the visitor with a vibration motor.
Should the visitor enter an exhibition to which they don't have access, the Dojo will vibrate and a red LED lights up.
% Results
In tests the Dojo can identify the closest of multiple beacons and can determine its exact distance with precision of about three meters.
It can store at least ten hours of Audio and up to 255 unique Beacon IDs and a complete key of 255 beacons can be written to the EEPROM in 6.17 seconds.
The battery works about eight hours before the deep discharge protection shuts the Dojo down.
% conclusions
The Dojo has shown itself to be a promising product, enhancing a visit to a museum.

\paragraph{Keywords:} Audio guide, Bluetooth, Bone conduction transducer, ATmega328p, WTV020, HM11

%\end{document}


\cleardoublepage
\include{00c_inhalt}

% ab hier ist diese Konvention für die Fusszeile eingestellt
\lfoot{Team 1\\Projekt 4}
\cfoot{\thepage}
\rfoot{\today}
\renewcommand{\footrulewidth}{0.4pt}

\chapter{Einleitung}
% ================ Einstellungen =======================
\thispagestyle{fancy} \rhead{\slshape Einleitung} \setcounter{page}{1}
% ======================================================
Museen bieten die Möglichkeit verschiedenste Ausstellung den Interessenten nahe zu bringen. Um dem Besucher zu ermöglichen, sich mit den persönlich relevanten Objekten auseinander zu setzen, werden unter anderem Audio-Guides verwendet. Die Audioguides werden meist mit Hilfe von Standard-Kopfhörern umgesetzt. Diese Kopfhörer müssen aufgrund von Hygienegründen gereinigt werden. Diese Reinigungskosten sind verhältnismässig zu gross. Zudem bieten die Museen nicht die Möglichkeit, eine Ausstellung einzugrenzen. Eine Unterteilung der Ausstellung würde dem Besucher die Option geben, eine Eintrittskarte zu erwerben, welche sich nur auf einen Teil der Ausstellung bezieht. Dies könnte Museumsaustellung attraktiver gestalten.\\

Ziel ist es, einen Audioguide zu entwickeln welcher es ermöglicht eine Ausstellung einzugrenzen, in dem es nur die gewählten Audiodateien abspielt und allenfalls als Zutrittsberechtigung fungiert. Die Audioübertragung soll über einen Knochenschallgeber funktionieren, dies würde die Reinigungskosten senken. Mit Hilfe des Audioguides soll es möglich sein, Sounddateien abzuspielen, sobald man sich in der Nähe eines Kunstobjektes befindet.\\

Mit Hilfe von Bluetooth Beacons können den Kunstobjekten Audiodateien zugeordnet werden. Durch einen Vibrator kann angekündigt werden, dass eine Audiodatei vorhanden ist, welche durch Bestätigung des Besuchers über den Knochenschallgeber abgespielt werden kann.\\

Mit Hilfe des Prototyps ist es möglich, Beacons in einem Abstand von xxx Metern korrekt zu erfassen und die zugeordnete Audiodatei abzuspielen. Auf dem Dojo können YYY Audiodateien abgespeichert werden. Am Eintritt können je nach Wunsch die verschiedenen Audiofiles ausgewählt werden und für den Besuch freigegeben werden. Dem Besucher wird die Möglichkeit gegeben, für gewünschte Objekte mit Hilfe eines Like-Buttons am Ende des Besuches Zusatzinformationen in Form einer Broschüre zu erhalten.\\

Der nachfolgende Bericht ist in sechs Teile aufgeteilt. Durch das Gesamtkonzept wird die Funktion sowie das Prinzip von Dojo erklärt. In den folgenden zwei Kapiteln wird die benötigte Energieversorgung sowie die Software für das Dojo dargelegt. In den letzten drei Kapiteln werden die Schnittstellen sowie die Firmware erläutert. 

\chapter{Gesamtsystem}
% ================ Einstellungen =======================
\thispagestyle{fancy} \rhead{\slshape Gesamtsystem}
% ======================================================



\definecolor{myviolett}{RGB}{212,192,226}
\definecolor{myyellow}{RGB}{255,255,178}
\definecolor{myorange}{RGB}{252,231,218}
\definecolor{mygreen}{RGB}{229,241,221}
\definecolor{myblue}{RGB}{225,237,247}
\definecolor{myred}{RGB}{255,178,178}


Abbildung \ref{Blockschaltbild_Gesamtsystem} zeigt das grobe Gesamtsystem des Dojos. Die einzelnen Kapitel dieses Berichts orientieren sich an den Blöcken und deren Funktion im Gesamtsystem. In Kapitel \textbf{\ref{Software} \nameref{Software}} wird die Computersoftware erklärt, welche verwendet wird, um das Dojo zu konfigurieren. Das Kapitel \textbf{\ref{Energieversorgung} \nameref{Energieversorgung}} befasst sich mit dem Akku, der Ladeschaltung und -überwachung, sowie der Spannungsversorgung. Im Kapitel \textbf{\ref{USB} \nameref{USB}} wird erläutert, wie über die USB-Schnittstelle mit dem Mikrocontroller kommuniziert wird und wie Audiofiles auf die $\mu$SD-Karte übertragen werden. Das Kapitel \textbf{\ref{Audioausgabe} \nameref{Audioausgabe}} behandelt die Verarbeitung und Ausgabe der Audiofiles. Das nächste Kapitel \textbf{\ref{Bluetooth} \nameref{Bluetooth}} befasst sich mit den Bluetooth-Komponenten, das heisst mit den Beacons zur Erkennung der Kunstobjekte und dem Bluetoothmodul im Dojo für den Empfang. Wie die Firmware als Gesamtsystem funktioniert, ist in Kapitel \textbf{\ref{Firmware} \nameref{Firmware}} dargelegt. Die Validierung der einzelnen Blöcke wird im jeweiligen Kapitel abgehandelt.

\begin{figure}[h]
	\centering
	\includegraphics[width=14cm]{Bilder/Gesamtsystem.png}
	\caption{Blockschaltbild Gesamtsystem}
	\label{Blockschaltbild_Gesamtsystem}
\end{figure}

\begin{table}[h]
	\centering
	\begin{tabular}{|c|c|c|c|c|c|} 
		\cellcolor{myviolett}Software & \cellcolor{myyellow}Energieversorgung & \cellcolor{myorange}USB & \cellcolor{mygreen}Audioausgabe & \cellcolor{myblue}Bluetooth & \cellcolor{myred}Firmware \\ 
	\end{tabular} 
	\caption{Legende Gesamtsystem}
	\label{legend_gesamtsystem}
\end{table}

\chapter{Software}
% ================ Einstellungen =======================
\thispagestyle{fancy} \rhead{\slshape Software}
% ======================================================
Im Kapitel Software wird erläutert, wie die Audiodateien auf das Dojo gelangen und die Korrespondenztabelle angepasst werden kann. Es wird erklärt, wie die Likes ausgewertet und verarbeitet werden können.
\section{Konzept}
In diesem unter Kapitel wir das erarbeitete Konzept für die Datenübertragung von einem Computer auf das Dojo dargelegt.

\section{Datenverwaltung}
Alle Austellungsobjete sind in einer Textdatei aufgelistet, in der vermerkt wird, welche Audio- und Textdatei zu welchem Beaon gehört.
Zudem wird in einer Weiteren Textdatei ein Preset definiert, mit der die SDKarte beschrieben werden kann.

Der CLI-Befehl nimmt die Inventardatei und die Presetdatei als Argumente und fragt den Nutzer, welche Ausstellungen und Sprache dem Besucher zur Ferfügung stehen sollen. 
Mit diesen Inputs wird ein Ticket generiert, das über die Serielle Schnittstelle auf das EEPROM des Mikrocontrollers geladen wird.

% In diesem Kapitel wird beschrieben wie die Konfigurationsdateien ausgelesen werden. Zudem wird die Korrespondenztabelle erläutert. 

\section{Auslesen, Schreiben}
Hier wird beschrieben wie die Verbindung und die Datenübertragung auf den internen EEPROM realisiert wird.

\section{Validierung}
Für die Validierung der Datenverwaltung wurden Python Unittests eingesetzt, die mit einer Testdatenbank alle Funktionen testen.
Neben den einzelnen Funktionen wird auch ein kompletter Durchlauf mit der Testdatenbank gemacht.

Die EEPROM-Opperationen werden mit einem Arduino Uno-Board getestet, wobei das EEPROM vom PC aus beschrieben und wieder eingelesen wird.
Die Unittest stellen sicher, dass das EEPROM die gewünschten Daten beinhaltet und dass diese mit der spezifizierten Geschwindigkeit übertragen werden.

% Hier werden die Python Unittests, welche verwendet wurden, dargelegt.


\chapter{Energieversorgung}
\label{Energieversorgung}
% ================ Einstellungen =======================
\thispagestyle{fancy} \rhead{\slshape Energieversorgung}
% ======================================================
\section{Technische Grundlagen}
In diesem unter Kapitel werden die technischen Grundlagen für das Verständnis der Energieversorgung dargelegt.
\section{Konzept}
Hier wird das verwendete Konzept der Energieversorgung respektive der Ladeschaltung erklärt und begründet. 
\section{Hardware}
Hier steht welche Bauteile in welcher Anordnung verwendet wurden.
\section{Validierung}
Hier wird erklärt, wie die Validierung der Ladeschaltung gelöst wurde. Die Resultate der Validierung (der Energieversorgung) und die eventuellen Abweichungen zu den Wünschen werden in diesem Kapitel beschrieben.

\chapter{USB}
\label{USB}
% ================ Einstellungen =======================
\thispagestyle{fancy} \rhead{\slshape USB} 
% ======================================================

Im folgenden Kapitel wird erklärt, wie über die USB-Schnittstelle Daten auf den Dojo geladen werden und ausgelesen werden. In einem Unterkapitel wird sogleich die Validierung beschrieben. 

\section{Technische Grundlagen}

Nachfolgend werden einige Begriffe zur Datenübertragung kurz erklärt, welche im Kapitel von Bedeutung sind.

\paragraph{UART}
Universal Asynchronous Receiver Transmitter (UART) realisiert eine digitale serielle Schnittstelle. Über die UART-Schnittstelle können Daten über einen Datenstrom gesendet und empfangen werden. Die Daten sind in einem fixen Rahmen, bestehend aus Start-Bit, fünf bis neun Datenbits, optionalem Parity-Bit zur Fehlererkennung und einem Stopp-Bit \cite{UARTdefinition}.

\paragraph{SDIO}
Secure Digital Input Output (SDIO) bezeichnet ein Interface für die Datenübertragung zwischen SD-Karten. Die Daten werden wahlweise im SPI, 1-Bit oder im 4-Bit Modus übertragen.

\section{Konzept}

Zur Kommunikation mit dem Mikrocontroller im Gerät besteht eine USB-UART-Schnittstelle. Um neue Audio-Dateien auf die $\mu$SD-Karte zu schreiben, wurde ausserdem eine USB-SDIO-Schnittstelle realisiert. Folgende Abbildung \ref{Konzept_USB} zeigt das Konzept mit den beiden Schnittstellen.

\begin{figure}[h]
	\centering
	\includegraphics[width=\textwidth]{Bilder/Konzept_USB.PNG}
	\caption{Konzept USB}
	\label{Konzept_USB}
\end{figure}

Eine Micro-USB Typ B Schnittstelle ist die gemeinsame Schnittstelle nach aussen. Dahinter schaltet ein USB-Switch, gesteuert durch den Mikrocontroller, einen der beiden Pfade durch. Es kann jeweils entweder der SDIO-Pfad oder UART-Pfad durchgeschaltet werden, nie beide Pfade gleichzeitig.\newline
Als USB-Switch wird der Baustein TS3USB30E von Texas Instruments verwendet. Dieser Chip wurde dafür entwickelt, um high-speed USB 2.0 Signale in portablen Geräten und Konsumelektronik zu schalten. Gesteuert wird der USB-Switch über zwei Pins. Über den Enable-Pin kann der Baustein in einen hochohmigen Zustand gebracht werden, um den Bus nicht zu belasten und den Stromverbrauch zu senken. Mit dem Select-Pin wird der durchzuschaltende Pfad ausgewählt \cite{DatasheetTS3USB}.

Folgende Tabelle \ref{truth_table_usb} zeigt die Wahrheitstabelle für die entsprechenden Pfade:
\begin{table}[h]
	\centering
	\begin{tabular}{|c|c|c|} 
		Select (Pin24) & $\overline{Enable}$ (Pin23) & Funktion \\ 
		\hline 
		X & HIGH & hochohmig \\ 
		\hline 
		LOW & LOW & Pfad = UART \\ 
		\hline 
		HIGH & LOW & Pfad = SDIO \\ 
	\end{tabular} 
	\caption{Wahrheitstabelle USB-Pfade}
	\label{truth_table_usb}
\end{table}

\section{USB zu UART}

Der Baustein FT231X von FTDI stellt das Interface von USB zu UART bereit. Das ganze USB-Protokoll wird dabei auf dem Chip abgehandelt und es sind kaum zusätzliche Bauteile nötig (siehe Schema im \nameref{Anhang}). Über zwei Pins (Rx und Tx) werden die Daten seriell übertragen.

Bei aktivem UART-Pfad wird der Dojo am Computer als serielle Schnittstelle erkannt. Über diese Schnittstelle kommuniziert nun die Software auf dem Computer mit dem Mikrocontroller im Gerät, um beispielsweise die \flq Likes\frq  auszulesen oder die Audiofileübertragung einzuleiten.


\section{USB zu SDIO}

Der VUB300 Chip vom Hersteller elan stellt das USB zu SDIO Interface bereit. Damit kann die $\mu$SD-Karte im Dojo über den USB-Anschluss beschrieben und ausgelesen werden. Wird mit dem USB-Switch der SDIO-Pfad durchgeschaltet, erscheint die $\mu$SD-Karte am Computer als Datenträger und kann gelesen sowie beschrieben werden. Zu beachten ist, dass zwischen dem VUB300 Chip und der $\mu$SD-Karte noch ein weiterer Multiplexer (MUX) eingebaut ist. Mit diesem wird die $\mu$SD-Karte zwischen dem VUB300 und dem Audio-Chip umgeschaltet. Genaueres dazu findet man in Abschnitt \textbf{\ref{Audioausgabe} \nameref{Audioausgabe}}.
Damit die $\mu$SD-Karte mit dem SDIO-Pfad verbunden ist, muss der MUX TS3A27518 (siehe Schema im \nameref{Anhang}) folgendermassen angesteuert werden:

\begin{table}[h]
	\centering
	\begin{tabular}{|c|c|c|c|} 
		$\overline{Enable}$ (Pin25) & IN1 (Pin26) & IN2 (Pin27) & Funktion \\ 
		\hline 
		HIGH & X & X & hochohmig \\ 
		\hline 
		LOW & LOW & LOW & Pfad = SDIO(VUB300) \\ 
		\hline 
		LOW & HIGH & HIGH & Pfad = Sound-Modul \\ 
	\end{tabular} 
	\caption{Wahrheitstabelle SD-Pfade}
	\label{truth_table_sd}
\end{table}

\section{Validierung UART-Pfad}

Um die korrekte Funktion der seriellen Schnittstelle zwischen dem USB-Anschluss und dem Mikrocontroller zu überprüfen, wurde eine kleine Testsoftware auf den Mikrocontroller geladen. Sobald der Mikrocontroller am RX-Pin Zeichen empfängt, sendet	er diese über den TX-Pin wieder zurück. Mit dem Serial Monitor der Arduino IDE wurden die Zeichenketten erfolgreich gesendet und empfangen.

\section{Validierung SDIO-Pfad}

Mit einem weiteren Testprogramm wurde die Funktionsfähigkeit des SDIO-Pfades überprüft. Dafür wurde der USB-Switch und der MUX mit dem Mikrocontroller entsprechend angesteuert. Die $\mu$SD-Karte wurde daraufhin von einem Computer mit Linux Betriebssystem als Datenträger erkannt und konnte erfolgreich ausgelesen und beschrieben werden. Leider stellte sich nach weiteren Tests heraus, dass auf einem Windows-Rechner die $\mu$SD-Karte nicht erkannt wird. Die Gründe dafür sind nicht bekannt und das Problem konnte auch nicht behoben werden. Somit muss die Computer-Software auf einem Linux Betriebssystem gestartet werden.
\paragraph{Anmerkung}$ $\\
Durch die begrenzte Verfügbarkeit des VUB300 Chips, sowie der geringen Informationsdichte über den Baustein und dem Kompatibilitätsproblem mit Windows, ist dieser nicht für weitere Projekte oder Serienproduktionen zu empfehlen.
\chapter{Audioausgabe}
\label{Audioausgabe}
% ================ Einstellungen =======================
\thispagestyle{fancy} \rhead{\slshape Audioausgabe} 
% ======================================================
In diesem Kapitel werden die technischen Grundlagen welche sich auf das verwendete Audio-Modul beziehen, das Konzept sowie die Funktion der Audioausgabe beschrieben. Zudem wird die Validierung erläutert.
\section{Technische Grundlagen}\label{TechWTV}
Das Konzept von Dojo sieht als Abspielgerät der Audiodateien einen Bone Conductor vor. Der Bone Conductor sowie das für den Prototyp verwendete Audio Modul WTV020 werden in diesem Kapitel beschrieben. Diese Informationen werden in den nachfolgenden Kapiteln benötigt. 
\paragraph{WTV020}
Das WTV020 Modul ist ein Soundmodul welches es ermöglicht, Audiodateien auf einem Aktor abzuspielen. Auf einer maximal 1GB grossen $\mu$SD-Karte können bis zu 512 Audiodateien abgespeichert werden. Die Audiodateien auf der $\mu$SD-Karte müssen jedoch dem .wav oder .ad4 Format entsprechen. Die Dateien müssen gemäss Vorgabe: 0000; 0001; 0002; … nummeriert werden.\\
Der WTV020 Chip kann in zwei verschiedenen Modes betrieben werden, dem MP3 Mode und dem Two Line Serial Mode. Im MP3 Mode können direkt 6 Pins angesteuert werden. Durch die sechs Pins können folgende Funktionen umgesetzt werden: Reset, $\pm$Volume , next, previous und play, pause. 
Für das Dojo wird jedoch der two line serial mode genützt. Dieser kann das Modul mit nur 3 Pins betreiben. Der Mikrocontroller muss an den Clock-, den Data- und den Busy-Pin angeschlossen werden. Im two line serial mode können die Audiodateien welche sich auf der $\mu$SD-Karte befinden abgespielt werden. Er ermöglicht zudem, ähnlich wie im MP3 Mode ein Lied zu pausieren und neu zu starten, sowie eine Lautstärkenregulation. \cite{WTV020}

\paragraph{Bone Conductor}
Die Audioausgabe erfolgt wie beschrieben über einen Bone Conductor. Der Bone Conductor besteht aus einem kleinen Metallstab, welcher mit einer Kupfer Spule umwickelt ist. Sobald ein pulsförmiger Strom durch die Spule fliest, dehnt sich ein Magnetfeld aus welches die benötigten Vibrationen auf ein flaches Metallstück auslöst. 
\begin{figure}[h]
	\centering
	\includegraphics[width=6cm]{Bilder/Bone-Conductor.jpg}
	\caption[Bone Conductor]{Bone Conductor \cite{BoneConductor}}
	\label{Bone-Conductor}
\end{figure}

In der Abbildung \ref{Bone-Conductor} ist der für den Dojo benötigten Bone Conductor abgebildet. Der Bone Conductor ermöglicht es durch die Vibrationen, dass eine Audio Datei über den Schädelknochen abgespielt wird und so nur für eine Person hörbar ist, diese jedoch immer noch die Umgebungsgeräusche wahrnimmt. \cite{BoneConductor}



\section{Konzept}\label{AudioKonzept}
Das Konzept der Audioausgabe ist wie folgt aufgebaut:
\begin{figure}[h]
	\centering
	\includegraphics[width=11cm]{Bilder/Audio-Konzept.jpg}
	\caption{Audio Konzept}
	\label{Audio-Konzept}
\end{figure}\\
Die auf einer $\mu$SD-Karte abgespeicherten Audiodateien werden über einen Multiplexer, welcher vom Mikrocontroller gesteuert wird, an einen WTV020 Chip übertragen. Dieser wird in einem Serial Mode \ref{TechWTV} betrieben und kann ebenfalls vom Mikrocontroller angesteuert werden. Der Audiochip entschlüsselt die Daten und gibt diese an einen Klasse D Verstärker weiter. Dieser wird benötigt, um eine gut hörbare Lautstärke zu erreichen. Das Audiosignal wird schliesslich an einem Bone Conductor ausgegeben. 
Nachfolgend wird die Anordnung auf dem Print der einzelnen Komponenten dargelegt. 
\section{Hardware}
Wie im Konzept \ref{AudioKonzept}
 beschrieben, muss man, um Daten von der $\mu$SD-Karte auszulesen zuerst den Multiplexer ansteuern. Dies geschieht über den Mikrocontroller. Im ungesteuerten Zustand ist der MUX auf den WTV020-Chip geschaltet. Der WTV020SD-20S Chip kann nun auf die $\mu$SD-Karte zugreifen. Über vier Pins werden die Audiofiles ausgelesen und weitergegeben. Das Audiosignal wird auf den Klasse-D Verstärker gegeben. Dieser ist wie folgt aufgebaut.
\begin{figure}[h!]
	\centering
	\includegraphics[width=10cm]{Bilder/Klasse-D.jpg}
	\caption{Klasse D Verstärker}
	\label{Klasse-D}
\end{figure}


Um ein Rauschen an der Speisung zu vermeiden, wird ein 1$\mu$ Farad grosser Kondensator benötigt. In diesem Projekt wurde zuerst eine konstante Verstärkung mit dem Faktor ca.10 verwendet, da die Lautstärken Regelung über den WTV020-Chip geregelt werden soll. Dieser Faktor entsteht aus dem Verhältnis des Internen Widerstandes und den zugeschalteten Widerständen. Um das Rauschen der Eingänge, sowie zu hohe Frequenzen zu filtern wurden zuerst wie die Abbildung \ref{Klasse-D} zeigt, Kapazitäten eingeplant. Auf dem Prototyp wurden aufgrund von Tests welche in der Validierung aufgeführt werden, die Vorwiderstände durch Kapazitäten ersetzt und die geplanten Kapazitäten weggelassen.\\ Wie erwähnt wird die Lautstärkenregelung über den WTV020-Chip geregelt. Dies ist jedoch im serial mode fehleranfällig. Übergangsweise wurde um die Lautstärke zu dämmen ein 50 Ohm grosser Widerstand vor den Bone Conductor geschaltet. Die Audioausgabe kann also wie beschrieben mit dem Bone Conductor umgesetzt werden.

\section{Firmware}
Wie im Kapitel Hardware beschrieben muss der MUX angesteuert werden um dem WTV-Chip den Zugriff auf die $\mu$SD-Karte zu gewähren. Im Kapitel \ref{USB} in der Tabelle \ref{truth_table_sd} ist beschrieben wie der MUX angesteuert werden muss, damit die benötigten Pins durchgeschaltet sind.
Für die Audioausgabe müssen alle Öffner geschlossen werden. Dies wird über die Pins 25 – 27 (Port C) realisiert. Sobald Musik abgespielt werden muss, werden die benötigten Ausgänge gesetzt. Das  WTV020 Modul vergleicht die Daten und Clock Pins wie folgt :
\begin{figure}[h]
	\centering
	\includegraphics[width=15cm]{Bilder/WTV-Serial-Mode.JPG}
	\caption{WTV Serial Mode}
	\label{WTV-Serial}
\end{figure}\\
In der Abbildung \ref{WTV-Serial} ist ein three line serial mode dargestellt. Für den benötigten two line serial mode wird der CS Pin nicht benötigt.
\newpage
Sobald am Pin 7 über einen Taster Play ein low Signal erkannt wird, wird folgende Funktion aufgerufen:

\begin{figure}[h]
	\begin{verbatim}
void sendWTVcommand(unsigned int command){
	digitalWrite(WTV_CLK, LOW);
	_delay_us(1900);
	for (byte i = 0; i < 16; i++)
	{
		_delay_us(100);
		digitalWrite(WTV_CLK, LOW);
		digitalWrite(WTV_DOUT, LOW);
		if ((command & 0x8000) != 0)
		{
			digitalWrite(WTV_DOUT, HIGH);
		}
		_delay_us(100);
		digitalWrite(WTV_CLK, HIGH);
		command = command<<1;
	}
}
	\end{verbatim}
	\caption[Audio Datei abspielen]{Audio Datei abspielen \cite{WTVCODE}}
	\label{WTV-Play}
\end{figure}


Die benötigten Befehle welche sendWTVcommand mitgegeben werden müssen sind aufgrund des WTV020-Chips vorgefertigt.
Nachfolgend werden die verwendeten Befehle und deren Funktion aufgeführt.\\

	
\begin{table}[h]
	\centering
	\begin{tabular}{|c|c|} 
		Funktion & Befehl\\ 
		\hline 
		Play\_Pause & 0xFFFE \\ 
		\hline 
		STOP & 0xFFFF \\ 
		\hline 
		VOL & 0xFFF0-0xFFF7 \\ 
	\end{tabular} 
	\caption{WTV020 Funktionen}
	\label{WTVFunktionen}
\end{table} 

Bei der Funktion Play\_Pause ist zu erwähnen, dass um die Ausgabe zu starten am Anfang die gewünschte File Nummer gesendet werden muss (0-512). Ohne File Nummer wird die aktuelle Audioausgabe pausiert und kann wieder gestartet werden.

\section{Validierung}
Um die Audioausgabe zu testen, wurden verschiedene Versuche durchgeführt. Zuerst wurde der Bone Conductor über eine kleine Verstärkerschaltung direkt an einem Laptop angeschlossen, um die benötigte Leistung und die Lautstärke abschätzen zu können. Bei diesen Messungen ergab sich, bei einer sehr gut hörbaren Lautstärke, eine maximale Scheinleistung von 0.28 VA.\\
Um das WTV020 Modul mit geringem Aufwand zu testen wurde dieses zuerst im MP3 Mode betrieben.
So konnten schon am Anfang nicht kompatible $\mu$SD-Karten aussortiert werden. \\
Alle Funktionen des WTV020-Chips wurden separat und im two line serial mode überprüft. Alle verlangten Funktionen des Chips laufen Wunschgemäss. Zudem wurde ein Versuchsaufbau mit dem D-Klasse Verstärker durchgeführt, mit welchem die aktuell maximal benötigte Leistung von 1W gemessen wurde. \\
Trotz der funktionierenden Tests ist aktuell am Prototyp die Lautstärkenregelung nicht möglich. Das Audiomodul übernimmt die eingestellte Lautstärke, setzt diese jedoch, sobald das Gerät an einen Verstärker angeschlossen ist, vorherigen Wert zurück. Zudem ist zu erwähnen, dass für eine Massenproduktion ein WTV020 Modul nicht geeignet wäre, aufgrund der erwähnten Eigenheiten. Das Modul nimmt trotz gleicher Formatierung und gleichem Hersteller nicht jede $\mu$SD-Karten an. Zudem treten beim serial mode immer wieder kleiner Ungereimtheiten auf, wie z.B. die Lautstärkenregelung. 

\chapter{Bluetooth}
\label{Bluetooth}
% ================ Einstellungen =======================
\thispagestyle{fancy} \rhead{\slshape Bluetooth}
% ======================================================
\section{Technische Grundlagen}
Da eine drahtlose Verbindung mit Beacons hergestellt werden muss, wird dazu ein Bluetooth-Modul, das HM-11, verwendet. Es ermöglicht das Übertragen der Identifikationsnummer vom Beacon auf das Dojo. Der Vorteil eines vorgefertigten Moduls liegt in der einfachen Ansteuerung Hardware-, sowie Firmware-mässig.

\subsection*{Technische Daten}
Das HM-11 Bluetooth-Modul benötigt eine Versorgungsspannung von 3.3 V DC und verfügt über das Kommunikationsprotokoll UART. Die Standard Baudrate beträgt 9600bps und nimmt beim Senden einen Strom von 15 mA auf. In offener Umgebung beträgt die Reichweite bis zu 30 m und unterstützt AT Kommandos, um das Bluetooth-Modul zu konfigurieren. 

\subsection*{Bluetooth-Konfiguration}
Damit der Prototyp die Beacons erkennt, wird ein Bluetooth-Modul, das HM-11, verwendet. Es ermöglicht das Übertragen der Identifikationsnummer vom Beacon auf die Printplatte. Um eine reibungslose Kommunikation zu gewährleisten, müssen einige Konfigurationen beim Bluetooth-Modul vorgenommen werden. Das Bluetooth-Modul wird mit einem FTDI verkabelt und am PC angeschlossen. Mit den AT-Kommandos wie AT+Name, AT+ROLE1, AT+IMME1 und AT+DISI? kann das Bluetooth-Modul nun eingestellt werden. Falls das Bluetooth-Modul bereits auf dem Print bestückt ist, kann das Modul über die UART-Schnittstelle konfiguriert werden. 

\section{Beacons im Museum}
Beacons sind kleine und kompakte Bluetooth-Sender die auf der Low Energy Spezifikation basieren. Sie können entweder batteriebetrieben oder mit permanentem Stromanschluss an Räumen oder Objekten installiert werden. Bewegt sich ein Museumsbesucher in die Nähe eines Beacons mit diesem Dojo, so kann er Audio- und Video-Beiträge empfangen und mit dem Dojo abspielen. Zudem kann an der Kasse des Museums entschieden werden, welche Zonen aktiviert werden sollen. Das Dojo beinhaltet somit auch die Zutrittsberechtigungen. Der Vorteil dieses Systems ist der Preis. Es wird nur so viel bezahlt, was den Kunden  interessiert. 

\subsection*{Minew E7}
Mit dem Beacon Minew E7 konnte ein Museum simuliert werden, um den Prototypen auf die Funktionalität zu prüfen. Dieser Beacon lässt sich über die gratis BeaconSET+ App konfigurieren. Dieser Beacon bringt einige Vorteile mit sich. Zum einen kriegt er eine maximale Reichweite von 100 m  hin, doch noch viel wichtiger ist das Protokoll. Der Minew E7 unterstützt iBeacon und Eddystone Protokolle die entweder separat oder gleichzeitig benutzt werden können. 

\subsection*{Protokollbeschreibung}
Das iBeacon-Protokoll wurde von Apple entwickelt und baut auf der Bluetooth-Low-Energy-Spezifikation auf. Das Paket das ausgestrahlt wird hat eine Länge von 30 Byte. In diesem Paket stecken die ganzen Informationen wie Geräte-ID, Beacon-Typ, Referenzempfangsleistung und noch viele weitere Daten drin. Um die Beacons zu scannen, wurde die Software so geschrieben, dass es nur die UUID, den Majorvalue und den RSSI-Wert benötogt. Nachfolgend wird der Aufbau des iBeacons-Paket graphisch dargestellt. 


%%%%%%
\begin{figure}[htp]
	\centering
	\includegraphics[width=15cm]{Bilder/iBeacon_Paket.PNG}
	 \caption{Aufbau eines iBeacon-Pakets}
	 \label{fig:iBeacon}
\end{figure}
%%%%%%

\subsection*{Receiver Signal Strength Indicator (RSSI)}
Der RSSI ist ein dimensionsloser Wert. Er beschreibt die empfangene Leistung bei kabelloser Kommunikation. Angegeben wird der Wert in dBm, was das Verhältnis in Dezibel zwischen der Leistung mW und der Referenzleistung 1 mW darstellt. 

\section{IBeacons auslesen und identifizieren}
Mittels der Funktion scan() im State SCAN wird mit dem hm-11 nach den verfügbaren IBeacons in der Umgebung gesucht. Über eine serielle Schnittstelle (UART) wird vom MCU der Befehl AT+DISI? dem hm-11 geschickt, wobei als Antwort dann alle umliegenden IBeacons in einem String zurückkommen (siehe Abbildung \ref{fig:disiCommand}).
\begin{figure}[h]
\centering
\includegraphics[scale=0.7]{Bilder/disi_command.PNG} 
\caption[Rückgabe des AT+DISI? Befehls]{P0: Factory ID (8 Byte); P1: IBeacon UUID (32 Byte); P2: Major-, Minorvalue, Measured Power (10 Byte); P3: Media-Access-Control-Adresse MAC (12 Byte); P4: RSSI (4 Byte) \cite{hm11Datasheet}}
\label{fig:disiCommand}
\end{figure}
Dafür wird vom MCU jeder char, rsp. jedes Byte vom Datenbuffer ausgelesen und verwertet. Es wird zuerst nach einer UUID eines IBeacons gefiltert und dann die ersten drei Zahlen in einen Integer gecastet. Anschließend soll das Majorvalue\footnote{Major- und Minorvalues dienen hauptsächlich zur zusätzlichen Identifikation eines IBeacons}, welches einen bestimmten, selbst wählbaren Wert hat, abgeglichen werden, damit nur die IBeacons berücksichtig werden, die relevant sind. Ist der empfangene RSSI-Wert grösser als -90dbm, wird der IBeacon eingespeichert. Dies wird mit jedem empfangenen IBeacon wiederholt und immer die RSSI-Werte miteinander verglichen, wobei dann der IBeacon mit dem grösseren RSSI-Wert ins System gespeichert und als Returnvalue von der scan() Funktion zurückgegeben wird.

\section{Validierung}
Für die Validierung wurden zwei IBeacons mit Androidhandys simuliert (Beacon Simulator App). Dafür wurden diese mit einer Distanz von ungefähr $6m$ zueinander auf einer Höhe von ca. $2m$ platziert. Um die detektierten IBeacons direkt auszuwerten, wurde die UUID auf einen Emulator (Putty) über eine serielle Schnittstelle (USB Typ micro B) herausgeschrieben. Somit konnte verifiziert werden, dass der vom Dojo detektierte IBeacon auch der sich am Dojo nächsten befindliche IBeacon war. 
\\[0.5cm]
Eine definitive Distanzbestimmung zwischen dem IBeacon und dem Dojo ist schwierig, da die Sendeleistung der simulierten IBeacons etwas schwanken. Auch die Auslegung des Raumes, sowie die Einrichtung kann das Signal abschwächen, was direkten Einfluss auf die Erkennungsdistanz hat. Es kann aber festgehalten werden, dass ein IBeacon in einem Umkreis von $3m$ garantiert erkennt wird, solange sich keine Gegenstände unmittelbar zwischen den beiden Objekten befinden.
\chapter{Firmware}
\label{Firmware}
% ================ Einstellungen =======================
\thispagestyle{fancy} \rhead{\slshape Firmware}
% ======================================================
\section{Konzept}
\section{Statemachine}
Auf dem Mikrocontroller läuft eine Mealy-Statemachine mit fünf Zuständen:\\
\begin{itemize}[leftmargin=3.2cm]
\item[SCAN:] Es wird vom Dojo nach IBeacons in naher Umgebung gesucht. Falls einer gefunden wurde, dann wird direkt das dazugehörige Audiofile bereitgestellt. Wird der Playbutton gedrückt, ändert sich der State zu PLAY. Ansonsten wird weiter gescanned.
\item[PLAY:] Hier werden die bereitgestellten Audiofiles abgespielt. Wird der Playbutton nochmals gedrückt, wird das Abspielen abgebrochen und der State wechselt wieder zu SCAN.
\end{itemize}
Wird das Dojo an den Computer angeschlossen, ist der Zustand des Dojos abhängig von der gewünschten Tätigkeit. Dafür wird vom Computer aus ein Befehl über das USB-Kabel gesendet, woraufhin sich der State ändert:
\begin{itemize}[leftmargin=3.2cm]
\item[GET\_Likes:] Die vom Besucher getätigten Likes werden vom EEPROM auf den Computer transferiert.
\item[LOAD\_SD:] Die microSD-Karte wird mit den gewünschten Audiofiles beschrieben.
\item[LOAD\_CONFIG:] Ein Ticket wird auf das interne EEPROM geladen.
\end{itemize}
 \section{Datenverwaltung}
In diesem Kapitel wird die Datenverwaltung auf dem internen EEPROM erklärt.
\section{Validierung}
Hier wird die Validierung der kompletten Firmware sowie dessen Ergebnisse dargelegt. 

\chapter{Schlusswort}
\label{Schlusswort}
% ================ Einstellungen =======================
\thispagestyle{fancy} \rhead{\slshape Schlusswort} 
% ======================================================

Der realisierte Prototyp ist in der Lage über BLE Beacons zu erkennen, eine entsprechende Rückmeldung über den Vibrationsmotor oder eine LED zu geben und ein zugehöriges Audiofile abzuspielen. Ausserdem kann die eingebaute $\mu$SD-Karte gelesen und beschrieben werden. Mit einem einfachen Command Line Interface, programmiert in Python, kann der Prototyp konfiguriert und \flq Likes\frq ausgelesen werden. Aus zeitlichen Gründen konnte die Übertragung des Konfigurationsfiles sowie die Auswertung der \flq Likes\frq noch nicht in das Gesamtsystem eingebettet werden. Das Konzept und die Software ist vorhanden, muss jedoch noch auf die Firmware des Mikrocontrollers abgestimmt werden.

Am Anfang des Projekts, wurde die USB Verbindung zur $\mu$SD-Karte, sowie die Umschaltung des Signals mit dem USB-Switch als sehr anspruchsvoll und fehleranfällig eingeschätzt. Durch sorgfältige Recherche und exaktem Layout wurden diese Hürden gut gemeistert. Durch die geringe Informationsmenge über den Audiochip WTV020 und seinen Eigenheiten, stellte sich seine Handhabung als anspruchsvoll heraus. Daraus resultierten einige Probleme im Verlaufe der Entwicklungsphase.

Bei einer allfälligen Weiterentwicklung des Produkts, sollte das Konzept überarbeitet werden. Eine Variante mit leistungsfähigerem Mikrocontroller bietet sich dabei an. Die Handhabung der $\mu$SD-Karte und Dekodierung der Audiofiles würde auf dem Mikrocontroller stattfinden. Dadurch kann das Kompatibilitätsproblem mit dem VUB300 Chip und Windows gelöst werden und man umgeht die Eigenheiten und Einschränkungen des WTV020 Soundchips.


\chapter{Verzeichnisse}
% ================ Einstellungen =======================
\thispagestyle{fancy} \rhead{\slshape Verzeichnisse}
% ======================================================
\makeatletter
\renewcommand\chapter{\thispagestyle{\chapterpagestyle}%
                    \global\@topnum\z@
                    \@afterindentfalse
                    \secdef\@chapter\@schapter}
\makeatother
\listoffigures
\thispagestyle{fancy}
\newpage
\bibliography{Literaturverzeichnis/lit_fachbericht}
\thispagestyle{fancy}

\chapter{Anhang}
% ================ Einstellungen =======================
\thispagestyle{fancy} \rhead{\slshape Anhang} 
% ============================================
\begin{figure}[h]
\centering
\includegraphics[scale=0.5]{Bilder/footprint_wtv020.png}
\label{fig:irgendesBild}
\caption[Abmessungen WTV020]{Abmessungen WTV020}
\end{figure}

Auf den folgenden Seiten befinden sich das Schema, das Layout und der Bestückungsplan der Printplatte.

\includepdf[pages=1,fitpaper]{pdfs/Schema_Spannungsversorgung}
\label{pdf:SchemaSpannungsversorgung}

\includepdf[pages=1,fitpaper]{pdfs/Schema_USB}
\label{pdf:SchemaUSB}

\includepdf[pages=1,fitpaper]{pdfs/Schema_Mikrocontroller}
\label{pdf:SchemaMikrocontroller}

\includepdf[pages=1,fitpaper]{pdfs/Bestueckung_Top}
\label{pdf:BestueckungTop}

\includepdf[pages=1,fitpaper]{pdfs/Bestueckung_Bottom}
\label{pdf:BestueckungBottom}

\includepdf[pages=1,fitpaper]{pdfs/Layout_all}
\label{pdf:LayoutAll}

\includepdf[pages=1,fitpaper]{pdfs/Layout_top}
\label{pdf:LayoutTop}

\includepdf[pages=1,fitpaper]{pdfs/Layout_bottom}
\label{pdf:LayoutBottom}

\end{document}


